
\begin{frame}{Collaboratori}

  \begin{itemize}
  \item Dott. Carlo Battilana, titolare modulo laboratorio
  \item Dott. Simone Rossi Tisbeni, tutor
  \end{itemize}

\end{frame}

\begin{frame}{Orario e ricevimento}
  \begin{itemize}
  \item Lezione/esercitazione: lunedì ore 16-18
  \item Laboratorio: indicativamente mercoledì 9-13, su due turni
    \begin{itemize}
    \item La partecipazione a Laboratorio è \textbf{obbligatoria}
    \end{itemize}
  \item Ricevimento su appuntamento
  \item Commenti, domande e suggerimenti sono benvenuti su chat (Teams) o via
    mail
  \end{itemize}
\end{frame}

\begin{frame}{Materiale di supporto}

  \begin{itemize}

  \item Questa presentazione, aggiornata a ogni lezione. L'ultima versione è
    disponibile a partire da \url{https://github.com/giacomini/pf2021}

  \item Come introduzione al \Cpp{}: B. Stroustrup, \textit{A tour of C++},
    \url{https://isocpp.org/tour}

  \item Come referenza online: \Cpp{} reference, \url{https://cppreference.com/}

  \item Testi consigliati (anche se non aggiornati alle ultime versioni del
    linguaggio)
    \begin{itemize}
    \item B.~Stroustrup,
      \href{https://stroustrup.com/programming.html}{\textit{Programming:
          Principles and Practice Using C++}}, 2nd edition, Addison-Wesley

    \item B.~Stroustrup, \href{https://stroustrup.com/4th.html}{\textit{The C++
          Programming Language}}, $4^{th}$ edition, Addison-Wesley

    \item B. Stroustrup, \textit{C++ -- Linguaggio, libreria standard, principi
        di programmazione}, IV edizione, Pearson
    \end{itemize}
  \end{itemize}
\end{frame}

\begin{frame}{Modalità d'esame}

  L'esame consiste in due prove:

  \begin{enumerate}

  \item Progetto riguardante l'implementazione di un programma \Cpp{}. Il progetto
    è svolto in parte durante le ore di laboratorio, in parte in autonomia. E'
    raccomandato lo svolgimento in gruppo.

    Maggiori dettagli verso metà corso.

  \item Colloquio orale riguardante la discussione del progetto e domande
    teoriche e pratiche sugli argomenti svolti a lezione.

    Al colloquio si accede solo con una valutazione sufficiente del progetto.

  \end{enumerate}

\end{frame}

% sondaggio
\begin{frame}
  \begin{center}
    \vfill
    \url{https://forms.office.com/r/EKNPDh2zsA}
    \vfill
    \includegraphics[height=.5\textheight]{images/sondaggio-qr.png}
    \vfill
  \end{center}
\end{frame}

\begin{frame}{Platforms, compilers, editors}
  \begin{itemize}[<+->]
  \item The reference platform is Linux (Ubuntu 20.04) with the \code{gcc}
    compiler suite
  \item But any platform with a recent compiler is fine, possibly with a
    Unix-like command-line shell (e.g. \code{bash} or \code{zsh})
  \item We'll provide some support to install and configure:
    \begin{itemize}[<.->]
    \item
      \href{https://github.com/giacomini/pf2021/blob/main/doc/WSLGuide.md}{Windows}:
      Ubuntu inside Window Subsystem for Linux
    \item
      \href{https://github.com/giacomini/pf2021/blob/main/doc/macOSGuide.md}{macOS}:
      XCode (-tools) with \code{gcc}
    \end{itemize}
  \item Any \textbf{textual} editor is fine
    \begin{itemize}[<.->]
    \item nano, vi, emacs, gedit, geany, notepad, \ldots
    \item we recommend \href{https://code.visualstudio.com/}{\textbf{Visual
          Studio Code}}
    \item LibreOffice Writer or MS Word are \textbf{not} text editors
    \end{itemize}
  \item You can also use an Integrated Development Environment (IDE)
    \begin{itemize}[<.->]
    \item Visual Studio, XCode, KDevelop, Eclipse, CLion, \ldots
    \end{itemize}
  \item Compilers online
    \begin{itemize}
    \item \url{https://godbolt.org/}
    \item \url{https://repl.it/}
    \end{itemize}
  \end{itemize}
\end{frame}

\begin{frame}{Course outline}
  \begin{itemize}
  \item<1-> Introduction to Linux/Unix
  \item<2-> Elements of computer architecture and operating systems
  \item<3-> Why \Cpp{}
  \item<4-> Objects, types, variables
  \item<4-> Expressions
  \item<4-> Statements and structured programming
  \item<4-> Functions
  \item<4-> User-defined types and classes
  \item<4-> Generic programming and templates
  \item<4-> The Standard Library, containers, algorithms
  \item<4-> Dynamic memory allocation
  \item<4-> Dynamic polymorphism (aka object-oriented programming)
  \item<4-> Error management
  \item<5-> Elements of software engineering and supporting tools
  \end{itemize}
\end{frame}

\begin{frame}{Prossimi appuntamenti}
  \begin{itemize}
  \item \textbf{Iscrivetevi al corso su virtuale}
  \item Mercoledì 22/9, due turni 9-11 e 11-13, Aula II Irnerio
    \begin{itemize}
    \item Installazione/configurazione portatili
    \item Prenotazione ``appello'' su AlmaEsami per tracciare presenze
    \end{itemize}

  \item Giovedì 23/9 e venerdì 24/9, ore 9-14, Aula II Irnerio
    \begin{itemize}
    \item Continuazione installazione/configurazione portatili
    \item Durante esame orale
    \end{itemize}

  \item Giovedì 30/9, ore 14-16, Lezione
  \item Venerdì 1/10, ore 14-16, Lezione
  \item Lunedì 11/10, ore 16-18, Lezione
  \item Mercoledì 13/10, ore 9-13, Introduzione a Linux/Unix
  \item Lunedì 18/10, ore 16-18, Lezione
  \item Mercoledì 20/10, ore 9-13, Laboratorio

  \end{itemize}
\end{frame}
